\documentclass[french]{article}

\usepackage[utf8]{inputenc}
\usepackage[T1]{fontenc}
\usepackage{babel}

\title{NumPy}

\begin{document}
\date{}

\maketitle


In Python, processing numbers is very slow. Numpy solves that. It's a very efficient numeric processing library. On top of that it's an array processing library.

\setlength{\parindent}{0cm}

\section{Basic NumPy Arrays}

\begin{verbatim}
  np.array(<list>)
\end{verbatim}

Rq. it's possible to have multi dimensional arrays\\

Get the array type:
\begin{verbatim}
  <array>.dtype
\end{verbatim}

Array dimensions:
\begin{verbatim}
  <array>.shape
\end{verbatim}

\section{Accessing the array}

Get a numPy array of selected indices:
\begin{verbatim}
  A[[i_1, i_2]]
\end{verbatim}

\subsection{$n=2$}
Access an element:
\begin{verbatim}
  A[i,j]
\end{verbatim}

It's possible to use slices:
\begin{verbatim}
  A[i_1:i_2,j_1:j_2]
\end{verbatim}

Get first row:
\begin{verbatim}
  A[0]
\end{verbatim}

You can modify a whole section of the array:
\begin{verbatim}
  A[0] = np.array(<new_row>)
\end{verbatim}

Rq. You can use the expansion (the whole row is set to 0):
\begin{verbatim}
  A[0] = 0
\end{verbatim}

\section{Summary statistics}

It possible to calculate some values:
\begin{verbatim}
  a.sum()
  a.mean()
  a.std()
  a.var()
\end{verbatim}

Rq. Also works for matrices (whole)\\

Matrices axes (or dimensions):
\begin{verbatim}
  A.sum(axis=0) get sum for each column
  A.sum(axis=1) get sum for each row
\end{verbatim}

\section{Broadcasting and Vectorized operations}

Get a linear array:
\begin{verbatim}
  np.arange(n)
\end{verbatim}

Vectorized operations are operations performed between both arrays and arrays, arrays and scalars which are extremely fast.\\

The opeartion will be applied to each one of the elements of the array:
\begin{verbatim}
  A + 10
\end{verbatim}
Rq. The operation is vectorized\\

Also possible to perform array, array operations:
\begin{verbatim}
  A + B
\end{verbatim}

\section{Boolean arrays}

Selection with booleans:
\begin{verbatim}
  A[[True, False, False, True]]
\end{verbatim}

These boolena arrays are the result of broadcasting boolean operations:
\begin{verbatim}
  A >= 2
\end{verbatim}

Which allows for:
\begin{verbatim}
  A[A >= 2]
\end{verbatim}

\section{Linear Algebra}

Dot product:
\begin{verbatim}
  A.dot(B)
\end{verbatim}

Cross product:
\begin{verbatim}
  A @ B
\end{verbatim}

Transpose
\begin{verbatim}
  A.T
\end{verbatim}

\section{Useful NumPy functions}

\subsection{random}
\begin{verbatim}
  np.random.random(size=2)
  np.random.normal(size=2)
  np.random.rand(2,4)
\end{verbatim}

\subsection{arange}
\begin{verbatim}
  np.arange(10)
  np.arange(5,10)
  np.arange(0,1,.1)
\end{verbatim}

\subsection{reshape}
\begin{verbatim}
  A.reshape()
\end{verbatim}

\subsection{linspace}
\begin{verbatim}
  A.linspace(0,1,5)
\end{verbatim}

\subsection{zeros, ones, empty}
\begin{verbatim}
  np.zeros((3,3))
  np.ones((3,3))
  np.empty((3,3))
\end{verbatim}

\end{document}
